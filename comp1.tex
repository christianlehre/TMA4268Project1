\documentclass[]{article}
\usepackage{lmodern}
\usepackage{amssymb,amsmath}
\usepackage{ifxetex,ifluatex}
\usepackage{fixltx2e} % provides \textsubscript
\ifnum 0\ifxetex 1\fi\ifluatex 1\fi=0 % if pdftex
  \usepackage[T1]{fontenc}
  \usepackage[utf8]{inputenc}
\else % if luatex or xelatex
  \ifxetex
    \usepackage{mathspec}
  \else
    \usepackage{fontspec}
  \fi
  \defaultfontfeatures{Ligatures=TeX,Scale=MatchLowercase}
\fi
% use upquote if available, for straight quotes in verbatim environments
\IfFileExists{upquote.sty}{\usepackage{upquote}}{}
% use microtype if available
\IfFileExists{microtype.sty}{%
\usepackage{microtype}
\UseMicrotypeSet[protrusion]{basicmath} % disable protrusion for tt fonts
}{}
\usepackage[margin=1in]{geometry}
\usepackage{hyperref}
\hypersetup{unicode=true,
            pdftitle={Compulsory exercise 1},
            pdfauthor={Christian Lehre, Axel R??nold \& Erik B??e},
            pdfborder={0 0 0},
            breaklinks=true}
\urlstyle{same}  % don't use monospace font for urls
\usepackage{color}
\usepackage{fancyvrb}
\newcommand{\VerbBar}{|}
\newcommand{\VERB}{\Verb[commandchars=\\\{\}]}
\DefineVerbatimEnvironment{Highlighting}{Verbatim}{commandchars=\\\{\}}
% Add ',fontsize=\small' for more characters per line
\usepackage{framed}
\definecolor{shadecolor}{RGB}{248,248,248}
\newenvironment{Shaded}{\begin{snugshade}}{\end{snugshade}}
\newcommand{\KeywordTok}[1]{\textcolor[rgb]{0.13,0.29,0.53}{\textbf{#1}}}
\newcommand{\DataTypeTok}[1]{\textcolor[rgb]{0.13,0.29,0.53}{#1}}
\newcommand{\DecValTok}[1]{\textcolor[rgb]{0.00,0.00,0.81}{#1}}
\newcommand{\BaseNTok}[1]{\textcolor[rgb]{0.00,0.00,0.81}{#1}}
\newcommand{\FloatTok}[1]{\textcolor[rgb]{0.00,0.00,0.81}{#1}}
\newcommand{\ConstantTok}[1]{\textcolor[rgb]{0.00,0.00,0.00}{#1}}
\newcommand{\CharTok}[1]{\textcolor[rgb]{0.31,0.60,0.02}{#1}}
\newcommand{\SpecialCharTok}[1]{\textcolor[rgb]{0.00,0.00,0.00}{#1}}
\newcommand{\StringTok}[1]{\textcolor[rgb]{0.31,0.60,0.02}{#1}}
\newcommand{\VerbatimStringTok}[1]{\textcolor[rgb]{0.31,0.60,0.02}{#1}}
\newcommand{\SpecialStringTok}[1]{\textcolor[rgb]{0.31,0.60,0.02}{#1}}
\newcommand{\ImportTok}[1]{#1}
\newcommand{\CommentTok}[1]{\textcolor[rgb]{0.56,0.35,0.01}{\textit{#1}}}
\newcommand{\DocumentationTok}[1]{\textcolor[rgb]{0.56,0.35,0.01}{\textbf{\textit{#1}}}}
\newcommand{\AnnotationTok}[1]{\textcolor[rgb]{0.56,0.35,0.01}{\textbf{\textit{#1}}}}
\newcommand{\CommentVarTok}[1]{\textcolor[rgb]{0.56,0.35,0.01}{\textbf{\textit{#1}}}}
\newcommand{\OtherTok}[1]{\textcolor[rgb]{0.56,0.35,0.01}{#1}}
\newcommand{\FunctionTok}[1]{\textcolor[rgb]{0.00,0.00,0.00}{#1}}
\newcommand{\VariableTok}[1]{\textcolor[rgb]{0.00,0.00,0.00}{#1}}
\newcommand{\ControlFlowTok}[1]{\textcolor[rgb]{0.13,0.29,0.53}{\textbf{#1}}}
\newcommand{\OperatorTok}[1]{\textcolor[rgb]{0.81,0.36,0.00}{\textbf{#1}}}
\newcommand{\BuiltInTok}[1]{#1}
\newcommand{\ExtensionTok}[1]{#1}
\newcommand{\PreprocessorTok}[1]{\textcolor[rgb]{0.56,0.35,0.01}{\textit{#1}}}
\newcommand{\AttributeTok}[1]{\textcolor[rgb]{0.77,0.63,0.00}{#1}}
\newcommand{\RegionMarkerTok}[1]{#1}
\newcommand{\InformationTok}[1]{\textcolor[rgb]{0.56,0.35,0.01}{\textbf{\textit{#1}}}}
\newcommand{\WarningTok}[1]{\textcolor[rgb]{0.56,0.35,0.01}{\textbf{\textit{#1}}}}
\newcommand{\AlertTok}[1]{\textcolor[rgb]{0.94,0.16,0.16}{#1}}
\newcommand{\ErrorTok}[1]{\textcolor[rgb]{0.64,0.00,0.00}{\textbf{#1}}}
\newcommand{\NormalTok}[1]{#1}
\usepackage{graphicx,grffile}
\makeatletter
\def\maxwidth{\ifdim\Gin@nat@width>\linewidth\linewidth\else\Gin@nat@width\fi}
\def\maxheight{\ifdim\Gin@nat@height>\textheight\textheight\else\Gin@nat@height\fi}
\makeatother
% Scale images if necessary, so that they will not overflow the page
% margins by default, and it is still possible to overwrite the defaults
% using explicit options in \includegraphics[width, height, ...]{}
\setkeys{Gin}{width=\maxwidth,height=\maxheight,keepaspectratio}
\IfFileExists{parskip.sty}{%
\usepackage{parskip}
}{% else
\setlength{\parindent}{0pt}
\setlength{\parskip}{6pt plus 2pt minus 1pt}
}
\setlength{\emergencystretch}{3em}  % prevent overfull lines
\providecommand{\tightlist}{%
  \setlength{\itemsep}{0pt}\setlength{\parskip}{0pt}}
\setcounter{secnumdepth}{0}
% Redefines (sub)paragraphs to behave more like sections
\ifx\paragraph\undefined\else
\let\oldparagraph\paragraph
\renewcommand{\paragraph}[1]{\oldparagraph{#1}\mbox{}}
\fi
\ifx\subparagraph\undefined\else
\let\oldsubparagraph\subparagraph
\renewcommand{\subparagraph}[1]{\oldsubparagraph{#1}\mbox{}}
\fi

%%% Use protect on footnotes to avoid problems with footnotes in titles
\let\rmarkdownfootnote\footnote%
\def\footnote{\protect\rmarkdownfootnote}

%%% Change title format to be more compact
\usepackage{titling}

% Create subtitle command for use in maketitle
\newcommand{\subtitle}[1]{
  \posttitle{
    \begin{center}\large#1\end{center}
    }
}

\setlength{\droptitle}{-2em}

  \title{Compulsory exercise 1}
    \pretitle{\vspace{\droptitle}\centering\huge}
  \posttitle{\par}
  \subtitle{TMA4268 Statistical Learning V2019}
  \author{Christian Lehre, Axel R??nold \& Erik B??e}
    \preauthor{\centering\large\emph}
  \postauthor{\par}
      \predate{\centering\large\emph}
  \postdate{\par}
    \date{19 February, 2019}


\begin{document}
\maketitle

\section{Problem 1: Multiple linear
regression}\label{problem-1-multiple-linear-regression}

\begin{Shaded}
\begin{Highlighting}[]
\KeywordTok{library}\NormalTok{(GLMsData)}
\KeywordTok{data}\NormalTok{(}\StringTok{"lungcap"}\NormalTok{)}
\NormalTok{lungcap}\OperatorTok{$}\NormalTok{Htcm=lungcap}\OperatorTok{$}\NormalTok{Ht}\OperatorTok{*}\FloatTok{2.54}
\NormalTok{modelA =}\StringTok{ }\KeywordTok{lm}\NormalTok{(}\KeywordTok{log}\NormalTok{(FEV) }\OperatorTok{~}\StringTok{ }\NormalTok{Age }\OperatorTok{+}\StringTok{ }\NormalTok{Htcm }\OperatorTok{+}\StringTok{ }\NormalTok{Gender }\OperatorTok{+}\StringTok{ }\NormalTok{Smoke, }\DataTypeTok{data=}\NormalTok{lungcap)}
\KeywordTok{summary}\NormalTok{(modelA)}
\end{Highlighting}
\end{Shaded}

\begin{verbatim}
## 
## Call:
## lm(formula = log(FEV) ~ Age + Htcm + Gender + Smoke, data = lungcap)
## 
## Residuals:
##      Min       1Q   Median       3Q      Max 
## -0.63278 -0.08657  0.01146  0.09540  0.40701 
## 
## Coefficients:
##              Estimate Std. Error t value Pr(>|t|)    
## (Intercept) -1.943998   0.078639 -24.721  < 2e-16 ***
## Age          0.023387   0.003348   6.984  7.1e-12 ***
## Htcm         0.016849   0.000661  25.489  < 2e-16 ***
## GenderM      0.029319   0.011719   2.502   0.0126 *  
## Smoke       -0.046067   0.020910  -2.203   0.0279 *  
## ---
## Signif. codes:  0 '***' 0.001 '**' 0.01 '*' 0.05 '.' 0.1 ' ' 1
## 
## Residual standard error: 0.1455 on 649 degrees of freedom
## Multiple R-squared:  0.8106, Adjusted R-squared:  0.8095 
## F-statistic: 694.6 on 4 and 649 DF,  p-value: < 2.2e-16
\end{verbatim}

\textbf{Q1:} \newline
Model A:
\[ \log(\text{FEV}) = \beta_0 + \beta_1AGE + \beta_2HTCM +\beta_3GENDERM  +
\beta_4SMOKE + \varepsilon\]

Fitted Model:
\[ \hat{\log(\text{FEV})} = -1.944 + 0.023AGE + 0.017HTCM + 0.029GENDERM -0.046SMOKE\]

\textbf{Q2:}

\begin{itemize}
\item
  \texttt{Estimate} - in particular interpretation of \texttt{Intercept}
  \newline  Estimated regression coefficients given by
  \[\hat{\mathbf{\beta}} = (\mathbf{X^T}\mathbf{X})^{-1}\mathbf{X}^T\mathbf{Y}\].
  \newline When increasing covariate \(x_j\) with one unit, and keeping
  all other covariates constant, the response variable changes with a
  factor of \(\hat{\beta_j}\). Positive estimates reduce the value of
  the response, while negative estimates increase the value. The
  intercept is the value of the response when all covariates are zero.
\item
  \texttt{Std.Error} \newline  Estimated standard deviation of the
  estimated regression coefficients, i.e the average amount that the
  estimated regression coefficients vary from the actual value. The
  Std.Error is given by
  \[\hat{\text{SD}}(\hat{\beta_j}) = \sqrt(\hat{\sigma}^2(\mathbf{X}^T\mathbf{X})^{-1}_{jj})\],
  where \(\mathbf{X}\) is the design matrix of the regression, and
  \(\hat{\sigma}\) is the estimated standard deviation of the response.
\item
  \texttt{Residual\ standard\ error} \newline
  Estimate of the standard deviation of the error term \(\epsilon\) in
  the regression model. The residual standard error is given by
  \[\hat{\sigma} = \frac{RSS}{n-p-1} = \frac{\sum_{i = 1}^n(Y_i - \hat{Y_i})^2}{n-p-1}\],
  where n is the number of observations and p is the number of
  covariates (or predictors) in the fitted model.
\item
  \texttt{F-statistic} \newline
  The F-statistic is used to test the hypothesis that all estimated
  regression coefficients are zero, i.e to check whether there is a
  relationship between response and predictors.
\end{itemize}

\textbf{Q3:} \newline  The proportion of variability of the model fit is
given by the \(R^2\)-statistic. The multiple \(R^2\) will always
increase with an increasing number of covariates, and might result in
too optimistic model assessment.The adjusted \(R^2\) takes this into
account, and adjust according to the number of covaraites. Thus, the
adjusted \(R^2\)-statistic is usually the preferred measure of explained
variability in the fitted model when doing multiple linear regression.

\textbf{Q4:}

\begin{Shaded}
\begin{Highlighting}[]
\KeywordTok{library}\NormalTok{(ggplot2)}
\CommentTok{# residuls vs fitted}
\KeywordTok{ggplot}\NormalTok{(modelA, }\KeywordTok{aes}\NormalTok{(.fitted, .stdresid)) }\OperatorTok{+}\StringTok{ }\KeywordTok{geom_point}\NormalTok{(}\DataTypeTok{pch =} \DecValTok{21}\NormalTok{) }\OperatorTok{+}
\StringTok{  }\KeywordTok{geom_hline}\NormalTok{(}\DataTypeTok{yintercept =} \DecValTok{0}\NormalTok{, }\DataTypeTok{linetype =} \StringTok{"dashed"}\NormalTok{) }\OperatorTok{+}
\StringTok{  }\KeywordTok{geom_smooth}\NormalTok{(}\DataTypeTok{se =} \OtherTok{FALSE}\NormalTok{, }\DataTypeTok{col =} \StringTok{"red"}\NormalTok{, }\DataTypeTok{size =} \FloatTok{0.5}\NormalTok{, }\DataTypeTok{method =} \StringTok{"loess"}\NormalTok{) }\OperatorTok{+}
\StringTok{  }\KeywordTok{labs}\NormalTok{(}\DataTypeTok{x =} \StringTok{"Fitted values"}\NormalTok{, }\DataTypeTok{y =} \StringTok{"Standardized residuals"}\NormalTok{,}
       \DataTypeTok{title =} \StringTok{"Fitted values vs. Standardized residuals"}\NormalTok{,}
       \DataTypeTok{subtitle =} \KeywordTok{deparse}\NormalTok{(modelA}\OperatorTok{$}\NormalTok{call))}
\end{Highlighting}
\end{Shaded}

\includegraphics{comp1_files/figure-latex/unnamed-chunk-2-1.pdf}

\begin{Shaded}
\begin{Highlighting}[]
\CommentTok{# qq-plot of residuals}
\KeywordTok{ggplot}\NormalTok{(modelA, }\KeywordTok{aes}\NormalTok{(}\DataTypeTok{sample =}\NormalTok{ .stdresid)) }\OperatorTok{+}
\StringTok{  }\KeywordTok{stat_qq}\NormalTok{(}\DataTypeTok{pch =} \DecValTok{19}\NormalTok{) }\OperatorTok{+}\StringTok{ }
\StringTok{  }\KeywordTok{geom_abline}\NormalTok{(}\DataTypeTok{intercept =} \DecValTok{0}\NormalTok{, }\DataTypeTok{slope =} \DecValTok{1}\NormalTok{, }\DataTypeTok{linetype =} \StringTok{"dotted"}\NormalTok{) }\OperatorTok{+}
\StringTok{  }\KeywordTok{labs}\NormalTok{(}\DataTypeTok{x =} \StringTok{"Theoretical quantiles"}\NormalTok{, }\DataTypeTok{y =} \StringTok{"Standardized residuals"}\NormalTok{, }
       \DataTypeTok{title =} \StringTok{"Normal Q-Q"}\NormalTok{, }\DataTypeTok{subtitle =} \KeywordTok{deparse}\NormalTok{(modelA}\OperatorTok{$}\NormalTok{call))}
\end{Highlighting}
\end{Shaded}

\includegraphics{comp1_files/figure-latex/unnamed-chunk-2-2.pdf}

\begin{Shaded}
\begin{Highlighting}[]
\CommentTok{# normality test}
\KeywordTok{library}\NormalTok{(nortest) }
\KeywordTok{ad.test}\NormalTok{(}\KeywordTok{rstudent}\NormalTok{(modelA))}
\end{Highlighting}
\end{Shaded}

\begin{verbatim}
## 
##  Anderson-Darling normality test
## 
## data:  rstudent(modelA)
## A = 1.9256, p-value = 6.486e-05
\end{verbatim}

In this problem we consider two diagnostic plots, namely residuals
vs.~fitted values and the Q-Q plot. The former plot showing residuals
vs.~fitted values looks good. The residuals are spread seemingly random,
and there is no evidence of a non-linear relationship between the
response and predictors. Thus, the model is good.

As for the Q-Q plot, the data quantiles seems to bend away from the
normal quantiles at the tails. That is, the residuals are heavy-tailed
rather than being noramlly distributed.

The normality test yields a low p-value, and the normality hypothesis is
rejected.

\textbf{Q5:}

\begin{Shaded}
\begin{Highlighting}[]
\NormalTok{modelB =}\StringTok{ }\KeywordTok{lm}\NormalTok{(FEV }\OperatorTok{~}\StringTok{ }\NormalTok{Age }\OperatorTok{+}\StringTok{ }\NormalTok{Htcm }\OperatorTok{+}\StringTok{ }\NormalTok{Gender }\OperatorTok{+}\StringTok{ }\NormalTok{Smoke, }\DataTypeTok{data=}\NormalTok{lungcap)}
\KeywordTok{summary}\NormalTok{(modelB)}

\CommentTok{#make diagnostic plots for modelB}
\CommentTok{# residuls vs fitted}
\KeywordTok{ggplot}\NormalTok{(modelB, }\KeywordTok{aes}\NormalTok{(.fitted, .stdresid)) }\OperatorTok{+}\StringTok{ }\KeywordTok{geom_point}\NormalTok{(}\DataTypeTok{pch =} \DecValTok{21}\NormalTok{) }\OperatorTok{+}
\StringTok{  }\KeywordTok{geom_hline}\NormalTok{(}\DataTypeTok{yintercept =} \DecValTok{0}\NormalTok{, }\DataTypeTok{linetype =} \StringTok{"dashed"}\NormalTok{) }\OperatorTok{+}
\StringTok{  }\KeywordTok{geom_smooth}\NormalTok{(}\DataTypeTok{se =} \OtherTok{FALSE}\NormalTok{, }\DataTypeTok{col =} \StringTok{"red"}\NormalTok{, }\DataTypeTok{size =} \FloatTok{0.5}\NormalTok{, }\DataTypeTok{method =} \StringTok{"loess"}\NormalTok{) }\OperatorTok{+}
\StringTok{  }\KeywordTok{labs}\NormalTok{(}\DataTypeTok{x =} \StringTok{"Fitted values"}\NormalTok{, }\DataTypeTok{y =} \StringTok{"Standardized residuals"}\NormalTok{,}
       \DataTypeTok{title =} \StringTok{"Fitted values vs. Standardized residuals"}\NormalTok{,}
       \DataTypeTok{subtitle =} \KeywordTok{deparse}\NormalTok{(modelB}\OperatorTok{$}\NormalTok{call))}
\end{Highlighting}
\end{Shaded}

\includegraphics{comp1_files/figure-latex/unnamed-chunk-3-1.pdf}

\begin{Shaded}
\begin{Highlighting}[]
\CommentTok{# qq-plot of residuals}
\KeywordTok{ggplot}\NormalTok{(modelB, }\KeywordTok{aes}\NormalTok{(}\DataTypeTok{sample =}\NormalTok{ .stdresid)) }\OperatorTok{+}
\StringTok{  }\KeywordTok{stat_qq}\NormalTok{(}\DataTypeTok{pch =} \DecValTok{19}\NormalTok{) }\OperatorTok{+}\StringTok{ }
\StringTok{  }\KeywordTok{geom_abline}\NormalTok{(}\DataTypeTok{intercept =} \DecValTok{0}\NormalTok{, }\DataTypeTok{slope =} \DecValTok{1}\NormalTok{, }\DataTypeTok{linetype =} \StringTok{"dotted"}\NormalTok{) }\OperatorTok{+}
\StringTok{  }\KeywordTok{labs}\NormalTok{(}\DataTypeTok{x =} \StringTok{"Theoretical quantiles"}\NormalTok{, }\DataTypeTok{y =} \StringTok{"Standardized residuals"}\NormalTok{, }
       \DataTypeTok{title =} \StringTok{"Normal Q-Q"}\NormalTok{, }\DataTypeTok{subtitle =} \KeywordTok{deparse}\NormalTok{(modelB}\OperatorTok{$}\NormalTok{call))}
\end{Highlighting}
\end{Shaded}

\includegraphics{comp1_files/figure-latex/unnamed-chunk-3-2.pdf}

\begin{Shaded}
\begin{Highlighting}[]
\CommentTok{# normality test}
\KeywordTok{library}\NormalTok{(nortest) }
\KeywordTok{ad.test}\NormalTok{(}\KeywordTok{rstudent}\NormalTok{(modelB))}
\end{Highlighting}
\end{Shaded}

\begin{verbatim}
## 
## Call:
## lm(formula = FEV ~ Age + Htcm + Gender + Smoke, data = lungcap)
## 
## Residuals:
##      Min       1Q   Median       3Q      Max 
## -1.37656 -0.25033  0.00894  0.25588  1.92047 
## 
## Coefficients:
##              Estimate Std. Error t value Pr(>|t|)    
## (Intercept) -4.456974   0.222839 -20.001  < 2e-16 ***
## Age          0.065509   0.009489   6.904 1.21e-11 ***
## Htcm         0.041023   0.001873  21.901  < 2e-16 ***
## GenderM      0.157103   0.033207   4.731 2.74e-06 ***
## Smoke       -0.087246   0.059254  -1.472    0.141    
## ---
## Signif. codes:  0 '***' 0.001 '**' 0.01 '*' 0.05 '.' 0.1 ' ' 1
## 
## Residual standard error: 0.4122 on 649 degrees of freedom
## Multiple R-squared:  0.7754, Adjusted R-squared:  0.774 
## F-statistic:   560 on 4 and 649 DF,  p-value: < 2.2e-16
## 
## 
##  Anderson-Darling normality test
## 
## data:  rstudent(modelB)
## A = 1.2037, p-value = 0.003853
\end{verbatim}

Looking at the residuals vs.~fitted values plot, one can see that the
spread is uneven and that there is a non-linear relationship between the
response and the covariates.

As for model fit B, the Smoke covariate is no longer significant, as its
corresponding p-value is relatively large. The coefficient of
determination, \(R^2\), which is a measure of the explained variability
of the model, is also lower for model B than model A. In other words,
model A explains a greater proportion of variability than model B.

The standard error, or estimated standard deviation of the estimated
regression coefficients, are lower in model A for all covarariates.
These quantities are used in e.g constructing confidence intervals for
the regression coefficients. As these are lower in model A, one will
obtain more accurate confidence intervals for \(\beta_j\) when using
model A.

When doing inference about FEV, the interpretation might be easier when
using model B, but the accuracy is greater when using model A.

\textbf{Q6:} Test if the covariate AGE has an effect on
\(\log(\text{FEV})\), i.e conduct the following hypothesis test \[
H_0: \beta_{\text{AGE}} = 0 \quad \text{vs.} \quad H_1: \beta_{\text{AGE}} \ne 0
\] using a two-sided t-test based on \[
T_{\text{AGE}} = \frac{\hat{\beta_{\text{AGE}}}-\beta_{\text{AGE}}}{\hat{SD}(\hat{\beta_{\text{AGE}}})} \sim t_{n-p-1} = t_{654-4-1}
\] where \(\hat{\text{SD}}(\hat{\beta_j})\) is the standard error as
explained in Q2 above.

\begin{Shaded}
\begin{Highlighting}[]
\CommentTok{#p-value AGE}
\NormalTok{p_AGE =}\StringTok{ }\KeywordTok{summary}\NormalTok{(modelA)}\OperatorTok{$}\NormalTok{coefficients[}\StringTok{'Age'}\NormalTok{,}\DecValTok{4}\NormalTok{]}

\CommentTok{#or equivalently}
\NormalTok{n =}\StringTok{ }\KeywordTok{dim}\NormalTok{(lungcap)[}\DecValTok{1}\NormalTok{]}
\NormalTok{p =}\StringTok{ }\DecValTok{4}
\NormalTok{df =}\StringTok{ }\NormalTok{n}\OperatorTok{-}\NormalTok{p}\OperatorTok{-}\DecValTok{1}
\NormalTok{alpha =}\StringTok{ }\FloatTok{0.05}
\NormalTok{critical_values =}\StringTok{ }\KeywordTok{c}\NormalTok{(}\KeywordTok{qt}\NormalTok{(alpha}\OperatorTok{/}\DecValTok{2}\NormalTok{,df),}\KeywordTok{qt}\NormalTok{(}\DecValTok{1}\OperatorTok{-}\NormalTok{alpha}\OperatorTok{/}\DecValTok{2}\NormalTok{,df)) }
\NormalTok{t_dist =}\StringTok{ }\KeywordTok{rt}\NormalTok{(}\DecValTok{10000}\NormalTok{,df)}

\NormalTok{t_stat =}\StringTok{ }\KeywordTok{abs}\NormalTok{(modelA}\OperatorTok{$}\NormalTok{coefficients[}\StringTok{'Age'}\NormalTok{]}\OperatorTok{/}\KeywordTok{summary}\NormalTok{(modelA)}\OperatorTok{$}\NormalTok{coefficients[}\StringTok{'Age'}\NormalTok{,}\StringTok{'Std. Error'}\NormalTok{])}

\CommentTok{#equivalently,}
\NormalTok{t_stat =}\StringTok{ }\KeywordTok{summary}\NormalTok{(modelA)}\OperatorTok{$}\NormalTok{coefficients[}\StringTok{'Age'}\NormalTok{,}\StringTok{'t value'}\NormalTok{]}

\CommentTok{#calculate p-value from t-statistic}
\NormalTok{p1_age =}\StringTok{ }\DecValTok{2}\OperatorTok{*}\KeywordTok{pt}\NormalTok{(}\OperatorTok{-}\KeywordTok{abs}\NormalTok{(t_stat),df)}

\KeywordTok{ggplot}\NormalTok{(}\KeywordTok{data.frame}\NormalTok{(}\DataTypeTok{x =}\NormalTok{ t_dist)) }\OperatorTok{+}\StringTok{ }\KeywordTok{geom_density}\NormalTok{(}\KeywordTok{aes}\NormalTok{(}\DataTypeTok{x =}\NormalTok{ x),}\DataTypeTok{fill =} \StringTok{'aliceblue'}\NormalTok{) }\OperatorTok{+}\StringTok{ }\KeywordTok{geom_vline}\NormalTok{(}\DataTypeTok{xintercept =} \KeywordTok{mean}\NormalTok{(t_dist) }\OperatorTok{+}\StringTok{ }\NormalTok{critical_values,}\DataTypeTok{color =} \StringTok{'red'}\NormalTok{) }\OperatorTok{+}\KeywordTok{geom_vline}\NormalTok{(}\DataTypeTok{xintercept =}\NormalTok{ t_stat,}\DataTypeTok{color =} \StringTok{'blue'}\NormalTok{) }\OperatorTok{+}\StringTok{ }\KeywordTok{labs}\NormalTok{(}\DataTypeTok{subtitle =} \KeywordTok{bquote}\NormalTok{(}\KeywordTok{list}\NormalTok{(alpha }\OperatorTok{==}\NormalTok{.(alpha),df }\OperatorTok{==}\NormalTok{.(df))),}\DataTypeTok{title =} \StringTok{'T-distribution'}\NormalTok{)}
\end{Highlighting}
\end{Shaded}

\includegraphics{comp1_files/figure-latex/unnamed-chunk-4-1.pdf} For a
given significance level \(\alpha\), the null-hypothesis is rejected if
the p-value is less than \(\alpha\). The p-value is

\begin{verbatim}
## [1] "P-value from summary of the fitted model:"
## [1] 7.09641e-12
## [1] "P-value computed from the t-statistic:"
## [1] 7.09641e-12
\end{verbatim}

Thus, the null-hypothesis is rejected for \(\alpha \ge 7\cdot10^{-12}\)
Observe in the above figure a student-t distribution with 649 degrees of
freedom, corresponding to the problem description. The critical values
for a significance level \(\alpha = 0.05\) is marked with red, vertical
lines, whereas the t-statistic assuming the null-hypothesis is maarked
with a blue line. The t-statistic is clearly within the rejection
region, and the null-hypothesis is rejected.

\textbf{Q7:} The \((1-\alpha)\%\) confidence interval (CI) for
\(\beta_{\text{Age}}\) for a given significance level \(\alpha\) is
given by

\[
[\hat{\beta}-t_{\frac{\alpha}{2},n-p-1}\hat{SD}(\hat{\beta}),\hat{\beta}+t_{\frac{\alpha}{2},n-p-1}\hat{SD}(\hat{\beta})]
\]

\begin{Shaded}
\begin{Highlighting}[]
\CommentTok{#99% confidence interval for beta_Age}
\NormalTok{betahat =}\StringTok{ }\KeywordTok{summary}\NormalTok{(modelA)}\OperatorTok{$}\NormalTok{coefficients[}\StringTok{'Age'}\NormalTok{,}\StringTok{'Estimate'}\NormalTok{]}
\NormalTok{sd =}\StringTok{ }\KeywordTok{summary}\NormalTok{(modelA)}\OperatorTok{$}\NormalTok{coefficients[}\StringTok{'Age'}\NormalTok{,}\StringTok{'Std. Error'}\NormalTok{]}
\NormalTok{ta2 =}\StringTok{ }\KeywordTok{qt}\NormalTok{(alpha}\OperatorTok{/}\DecValTok{2}\NormalTok{, df)}
\NormalTok{UCI =}\StringTok{ }\NormalTok{betahat }\OperatorTok{+}\StringTok{ }\NormalTok{ta2}\OperatorTok{*}\NormalTok{sd}
\NormalTok{LCI =}\StringTok{ }\NormalTok{betahat }\OperatorTok{-}\StringTok{ }\NormalTok{ta2}\OperatorTok{*}\NormalTok{sd}
\NormalTok{CI =}\StringTok{ }\KeywordTok{c}\NormalTok{(UCI,LCI)}
\CommentTok{#or, equivalently}
\KeywordTok{print}\NormalTok{(}\StringTok{"built-in CI:"}\NormalTok{)}
\KeywordTok{confint}\NormalTok{(modelA,}\StringTok{'Age'}\NormalTok{,}\DataTypeTok{level =} \FloatTok{0.99}\NormalTok{)}
\KeywordTok{print}\NormalTok{(}\StringTok{" "}\NormalTok{)}
\KeywordTok{print}\NormalTok{(}\StringTok{"Numerically from formula:"}\NormalTok{)}
\NormalTok{CI}
\end{Highlighting}
\end{Shaded}

\begin{verbatim}
## [1] "built-in CI:"
##          0.5 %     99.5 %
## Age 0.01473674 0.03203769
## [1] " "
## [1] "Numerically from formula:"
## [1] 0.01681211 0.02996232
\end{verbatim}

Since the \(99\%\) CI for \(\beta_{\text{Age}}\) does not include \(0\),
the p-value is less than \(\alpha = 0.01\).

\textbf{Q8:} Prediction of a 16 year old male that is 170cm and
non-smoking

\begin{Shaded}
\begin{Highlighting}[]
\NormalTok{new =}\StringTok{ }\KeywordTok{data.frame}\NormalTok{(}\DataTypeTok{Age=}\DecValTok{16}\NormalTok{, }\DataTypeTok{Htcm=}\DecValTok{170}\NormalTok{, }\DataTypeTok{Gender=}\StringTok{"M"}\NormalTok{, }\DataTypeTok{Smoke=}\DecValTok{0}\NormalTok{)}
\NormalTok{pred =}\StringTok{ }\KeywordTok{predict}\NormalTok{(modelA,}\DataTypeTok{newdata =}\NormalTok{ new, }\DataTypeTok{type =} \StringTok{'response'}\NormalTok{)}
\CommentTok{# Construct 95% prediction interval for FEV, note that modelA's response is log(FEV)}
\NormalTok{pred}
\NormalTok{f.exp =}\StringTok{ }\ControlFlowTok{function}\NormalTok{(x)\{}
  \KeywordTok{return}\NormalTok{ (}\KeywordTok{exp}\NormalTok{(x))}
\NormalTok{\}}

\NormalTok{fev.pred =}\StringTok{ }\KeywordTok{f.exp}\NormalTok{(pred)}

\NormalTok{fev.predint =}\StringTok{ }\KeywordTok{predict}\NormalTok{(modelA,}\DataTypeTok{newdata =}\NormalTok{ new, }\DataTypeTok{level =} \FloatTok{0.95}\NormalTok{, }\DataTypeTok{interval =} \StringTok{'prediction'}\NormalTok{)}
\KeywordTok{f.exp}\NormalTok{(fev.predint)}
\end{Highlighting}
\end{Shaded}

\begin{verbatim}
##        1 
## 1.323802 
##       fit      lwr      upr
## 1 3.75768 2.818373 5.010038
\end{verbatim}

Best prediction for \(\log(\text{FEV})\) is 1.324. The \(95\%\)
prediction interval for the forced expiratory volume, FEV, of the new
observation is \([2.812, 5.010]\). The prediction interval is too wide
to be useful. A person with the given characteristic will on average
have a \(95\%\) chance of having a FEV between 2.8 and 5.0

\section{Problem 2: Classification}\label{problem-2-classification}

\begin{Shaded}
\begin{Highlighting}[]
\KeywordTok{library}\NormalTok{(class)}\CommentTok{# for function knn}
\KeywordTok{library}\NormalTok{(caret)}\CommentTok{# for confusion matrices}
\end{Highlighting}
\end{Shaded}

\begin{verbatim}
## Loading required package: lattice
\end{verbatim}

\begin{Shaded}
\begin{Highlighting}[]
\KeywordTok{library}\NormalTok{(ggplot2)}\CommentTok{# for ggplot}
\NormalTok{raw =}\StringTok{ }\KeywordTok{read.csv}\NormalTok{(}\StringTok{"https://www.math.ntnu.no/emner/TMA4268/2019v/data/tennis.csv"}\NormalTok{)}
\NormalTok{M =}\StringTok{ }\KeywordTok{na.omit}\NormalTok{(}\KeywordTok{data.frame}\NormalTok{(}\DataTypeTok{y=}\KeywordTok{as.factor}\NormalTok{(raw}\OperatorTok{$}\NormalTok{Result),}
                       \DataTypeTok{x1=}\NormalTok{raw}\OperatorTok{$}\NormalTok{ACE.}\DecValTok{1}\OperatorTok{-}\NormalTok{raw}\OperatorTok{$}\NormalTok{UFE.}\DecValTok{1}\OperatorTok{-}\NormalTok{raw}\OperatorTok{$}\NormalTok{DBF.}\DecValTok{1}\NormalTok{, }
                       \DataTypeTok{x2=}\NormalTok{raw}\OperatorTok{$}\NormalTok{ACE.}\DecValTok{2}\OperatorTok{-}\NormalTok{raw}\OperatorTok{$}\NormalTok{UFE.}\DecValTok{2}\OperatorTok{-}\NormalTok{raw}\OperatorTok{$}\NormalTok{DBF.}\DecValTok{2}\NormalTok{))}
\KeywordTok{set.seed}\NormalTok{(}\DecValTok{4268}\NormalTok{) }\CommentTok{# for reproducibility}
\NormalTok{tr =}\StringTok{ }\KeywordTok{sample.int}\NormalTok{(}\KeywordTok{nrow}\NormalTok{(M),}\KeywordTok{nrow}\NormalTok{(M)}\OperatorTok{/}\DecValTok{2}\NormalTok{)}
\NormalTok{trte=}\KeywordTok{rep}\NormalTok{(}\DecValTok{1}\NormalTok{,}\KeywordTok{nrow}\NormalTok{(M))}
\NormalTok{trte[tr]=}\DecValTok{0}
\NormalTok{Mdf=}\KeywordTok{data.frame}\NormalTok{(M,}\StringTok{"istest"}\NormalTok{=}\KeywordTok{as.factor}\NormalTok{(trte))}
\KeywordTok{ggplot}\NormalTok{(}\DataTypeTok{data=}\NormalTok{Mdf,}\KeywordTok{aes}\NormalTok{(}\DataTypeTok{x=}\NormalTok{x1,}\DataTypeTok{y=}\NormalTok{x2,}\DataTypeTok{colour=}\NormalTok{y,}\DataTypeTok{group=}\NormalTok{istest,}\DataTypeTok{shape=}\NormalTok{istest))}\OperatorTok{+}
\StringTok{  }\KeywordTok{geom_point}\NormalTok{()}\OperatorTok{+}\KeywordTok{theme_minimal}\NormalTok{()}
\end{Highlighting}
\end{Shaded}

\includegraphics{comp1_files/figure-latex/unnamed-chunk-8-1.pdf}

\textbf{Q9:} The mathematical formula for the K-nearest neighbour
esitimator \(\hat{y} \in \{0,1\}\) is given by \[
\hat{P}(Y = j | X = x_0) = \frac{1}{K}\sum_{i\in N_0}I(y_i = j)
\] where K is the flexibility parameter determining how many points are
in the neighbouring set \(N_0\). The classifier identifies the K nearest
points to a test observation \(x_0\) in the training data, and
calculates the conditional, or posterior, probability of being in class
\(j\) as the fraction of points in \(N_0\) where the response is \(j\).
\(I(y_i = j)\) is a binomial function yielding 1 if \(y_i = j\), and 0
elsewhere.

\textbf{Q10:}

\begin{Shaded}
\begin{Highlighting}[]
\CommentTok{# Misclassification error for training and test data using Knn with k = 1:30}
\NormalTok{knn.train =}\StringTok{ }\NormalTok{Mdf[tr,]}
\NormalTok{knn.test =}\StringTok{ }\NormalTok{Mdf[}\OperatorTok{-}\NormalTok{tr,]}

\NormalTok{K =}\StringTok{ }\DecValTok{30}
\NormalTok{train.error =}\StringTok{ }\KeywordTok{rep}\NormalTok{(}\OtherTok{NA}\NormalTok{,K)}
\NormalTok{test.error =}\StringTok{ }\KeywordTok{rep}\NormalTok{(}\OtherTok{NA}\NormalTok{,K)}
\ControlFlowTok{for}\NormalTok{ (k }\ControlFlowTok{in} \DecValTok{1}\OperatorTok{:}\NormalTok{K)\{}
\NormalTok{  test.pred =}\StringTok{ }\NormalTok{class}\OperatorTok{::}\KeywordTok{knn}\NormalTok{(}\DataTypeTok{train =}\NormalTok{ knn.train,}\DataTypeTok{test =}\NormalTok{ knn.test,}\DataTypeTok{cl =}\NormalTok{ knn.train}\OperatorTok{$}\NormalTok{y,}\DataTypeTok{k =}\NormalTok{ k)}
\NormalTok{  train.pred =}\StringTok{ }\NormalTok{class}\OperatorTok{::}\KeywordTok{knn}\NormalTok{(}\DataTypeTok{train =}\NormalTok{ knn.train, }\DataTypeTok{test =}\NormalTok{ knn.train, }\DataTypeTok{cl =}\NormalTok{ knn.train}\OperatorTok{$}\NormalTok{y, }\DataTypeTok{k =}\NormalTok{k)}
\NormalTok{  test.error[k] =}\StringTok{ }\KeywordTok{mean}\NormalTok{(test.pred }\OperatorTok{!=}\StringTok{ }\NormalTok{knn.test}\OperatorTok{$}\NormalTok{y)}
\NormalTok{  train.error[k] =}\StringTok{ }\KeywordTok{mean}\NormalTok{(train.pred }\OperatorTok{!=}\StringTok{ }\NormalTok{knn.train}\OperatorTok{$}\NormalTok{y)}
\NormalTok{\}}
\NormalTok{train.e =}\StringTok{ }\NormalTok{train.error}
\NormalTok{test.e =}\StringTok{ }\NormalTok{test.error}

\NormalTok{train.error.df =}\StringTok{ }\KeywordTok{data.frame}\NormalTok{(}\DataTypeTok{k=}\DecValTok{1}\OperatorTok{:}\NormalTok{K, }\DataTypeTok{error =}\NormalTok{ train.error)}
\KeywordTok{ggplot}\NormalTok{(train.error.df, }\KeywordTok{aes}\NormalTok{(}\DataTypeTok{x=}\NormalTok{k, }\DataTypeTok{y=}\NormalTok{error))}\OperatorTok{+}\KeywordTok{geom_point}\NormalTok{(}\DataTypeTok{col=}\StringTok{"blue"}\NormalTok{)}\OperatorTok{+}\KeywordTok{geom_line}\NormalTok{(}\DataTypeTok{linetype=}\StringTok{"dotted"}\NormalTok{) }\OperatorTok{+}\StringTok{ }\KeywordTok{ggtitle}\NormalTok{(}\StringTok{'Misclassification rates KNN - Training data'}\NormalTok{)}
\end{Highlighting}
\end{Shaded}

\includegraphics{comp1_files/figure-latex/unnamed-chunk-9-1.pdf}

\begin{Shaded}
\begin{Highlighting}[]
\NormalTok{test.error.df =}\StringTok{ }\KeywordTok{data.frame}\NormalTok{(}\DataTypeTok{k=}\DecValTok{1}\OperatorTok{:}\NormalTok{K, }\DataTypeTok{error =}\NormalTok{ test.error)}
\KeywordTok{ggplot}\NormalTok{(test.error.df, }\KeywordTok{aes}\NormalTok{(}\DataTypeTok{x=}\NormalTok{k, }\DataTypeTok{y=}\NormalTok{error))}\OperatorTok{+}\KeywordTok{geom_point}\NormalTok{(}\DataTypeTok{col=}\StringTok{"blue"}\NormalTok{)}\OperatorTok{+}\KeywordTok{geom_line}\NormalTok{(}\DataTypeTok{linetype=}\StringTok{"dotted"}\NormalTok{) }\OperatorTok{+}\StringTok{ }\KeywordTok{ggtitle}\NormalTok{(}\StringTok{'Misclassification rates KNN - Test data'}\NormalTok{)}
\end{Highlighting}
\end{Shaded}

\includegraphics{comp1_files/figure-latex/unnamed-chunk-9-2.pdf}

\begin{Shaded}
\begin{Highlighting}[]
\KeywordTok{set.seed}\NormalTok{(}\DecValTok{0}\NormalTok{)}
\NormalTok{ks =}\StringTok{ }\DecValTok{1}\OperatorTok{:}\DecValTok{30} \CommentTok{# Choose K from 1 to 30.}
\NormalTok{idx =}\StringTok{ }\KeywordTok{createFolds}\NormalTok{(M[tr,}\DecValTok{1}\NormalTok{], }\DataTypeTok{k=}\DecValTok{5}\NormalTok{) }\CommentTok{# Divide the training data into 5 folds.}
\CommentTok{# "Sapply" is a more efficient for-loop. }
\CommentTok{# We loop over each fold and each value in "ks"}
\CommentTok{# and compute error rates for each combination.}
\CommentTok{# All the error rates are stored in the matrix "cv", }
\CommentTok{# where folds are rows and values of $K$ are columns.}
\NormalTok{cv =}\StringTok{ }\KeywordTok{sapply}\NormalTok{(ks, }\ControlFlowTok{function}\NormalTok{(k)\{ }
  \KeywordTok{sapply}\NormalTok{(}\KeywordTok{seq_along}\NormalTok{(idx), }\ControlFlowTok{function}\NormalTok{(j) \{}
\NormalTok{    yhat =}\StringTok{ }\NormalTok{class}\OperatorTok{::}\KeywordTok{knn}\NormalTok{(}\DataTypeTok{train=}\NormalTok{M[tr[ }\OperatorTok{-}\NormalTok{idx[[j]] ], }\OperatorTok{-}\DecValTok{1}\NormalTok{],}
               \DataTypeTok{cl=}\NormalTok{M[tr[ }\OperatorTok{-}\NormalTok{idx[[j]] ], }\DecValTok{1}\NormalTok{],}
               \DataTypeTok{test=}\NormalTok{M[tr[ idx[[j]] ], }\OperatorTok{-}\DecValTok{1}\NormalTok{], }\DataTypeTok{k =}\NormalTok{ k)}
    \KeywordTok{mean}\NormalTok{(M[tr[ idx[[j]] ], }\DecValTok{1}\NormalTok{] }\OperatorTok{!=}\StringTok{ }\NormalTok{yhat)}
\NormalTok{  \})}
\NormalTok{\})}
\end{Highlighting}
\end{Shaded}

\textbf{Q11:}

\begin{Shaded}
\begin{Highlighting}[]
\NormalTok{cv.e =}\StringTok{ }\KeywordTok{colMeans}\NormalTok{(cv)}
\NormalTok{cv.se =}\StringTok{ }\KeywordTok{apply}\NormalTok{(cv,}\DecValTok{2}\NormalTok{,sd)}
\NormalTok{k.min =}\StringTok{ }\KeywordTok{which.min}\NormalTok{(cv.e)}
\end{Highlighting}
\end{Shaded}

\textbf{Q12:}

\begin{Shaded}
\begin{Highlighting}[]
\KeywordTok{library}\NormalTok{(colorspace)}
\NormalTok{co =}\StringTok{ }\KeywordTok{rainbow_hcl}\NormalTok{(}\DecValTok{3}\NormalTok{)}
\KeywordTok{par}\NormalTok{(}\DataTypeTok{mar=}\KeywordTok{c}\NormalTok{(}\DecValTok{4}\NormalTok{,}\DecValTok{4}\NormalTok{,}\DecValTok{1}\NormalTok{,}\DecValTok{1}\NormalTok{)}\OperatorTok{+}\NormalTok{.}\DecValTok{1}\NormalTok{, }\DataTypeTok{mgp =} \KeywordTok{c}\NormalTok{(}\DecValTok{3}\NormalTok{, }\DecValTok{1}\NormalTok{, }\DecValTok{0}\NormalTok{))}
\KeywordTok{plot}\NormalTok{(ks, cv.e, }\DataTypeTok{type=}\StringTok{"o"}\NormalTok{, }\DataTypeTok{pch =} \DecValTok{16}\NormalTok{, }\DataTypeTok{ylim =} \KeywordTok{c}\NormalTok{(}\DecValTok{0}\NormalTok{, }\FloatTok{0.7}\NormalTok{), }\DataTypeTok{col =}\NormalTok{ co[}\DecValTok{2}\NormalTok{],}
     \DataTypeTok{xlab =} \StringTok{"Number of neighbors"}\NormalTok{, }\DataTypeTok{ylab=}\StringTok{"Misclassification error"}\NormalTok{)}
\KeywordTok{arrows}\NormalTok{(ks, cv.e}\OperatorTok{-}\NormalTok{cv.se, ks, cv.e}\OperatorTok{+}\NormalTok{cv.se, }\DataTypeTok{angle=}\DecValTok{90}\NormalTok{, }\DataTypeTok{length=}\NormalTok{.}\DecValTok{03}\NormalTok{, }\DataTypeTok{code=}\DecValTok{3}\NormalTok{, }\DataTypeTok{col=}\NormalTok{co[}\DecValTok{2}\NormalTok{])}
\KeywordTok{lines}\NormalTok{(ks, train.e, }\DataTypeTok{type=}\StringTok{"o"}\NormalTok{, }\DataTypeTok{pch =} \DecValTok{16}\NormalTok{, }\DataTypeTok{ylim =} \KeywordTok{c}\NormalTok{(}\FloatTok{0.5}\NormalTok{, }\FloatTok{0.7}\NormalTok{), }\DataTypeTok{col =}\NormalTok{ co[}\DecValTok{3}\NormalTok{])}
\KeywordTok{lines}\NormalTok{(ks, test.e, }\DataTypeTok{type=}\StringTok{"o"}\NormalTok{, }\DataTypeTok{pch =} \DecValTok{16}\NormalTok{, }\DataTypeTok{ylim =} \KeywordTok{c}\NormalTok{(}\FloatTok{0.5}\NormalTok{, }\FloatTok{0.7}\NormalTok{), }\DataTypeTok{col =}\NormalTok{ co[}\DecValTok{1}\NormalTok{])}
\KeywordTok{legend}\NormalTok{(}\StringTok{"topright"}\NormalTok{, }\DataTypeTok{legend =} \KeywordTok{c}\NormalTok{(}\StringTok{"Test"}\NormalTok{, }\StringTok{"5-fold CV"}\NormalTok{, }\StringTok{"Training"}\NormalTok{), }\DataTypeTok{lty =} \DecValTok{1}\NormalTok{, }\DataTypeTok{col=}\NormalTok{co)}
\end{Highlighting}
\end{Shaded}

\includegraphics{comp1_files/figure-latex/unnamed-chunk-12-1.pdf} In
K-nearest neighbour, the degree of flexibility decrease with K. That is,
the lower the K, the lower the bias. Therefore, the bias increase with
K. According to the bias-variance tradeoff, the variance will decrease
with K.

\textbf{Q13:}

\begin{Shaded}
\begin{Highlighting}[]
\NormalTok{k =}\StringTok{ }\KeywordTok{tail}\NormalTok{(}\KeywordTok{which}\NormalTok{(cv.e }\OperatorTok{<}\StringTok{ }\NormalTok{cv.e[k.min] }\OperatorTok{+}\StringTok{ }\NormalTok{cv.se[k.min]), }\DecValTok{1}\NormalTok{)}
\NormalTok{size =}\StringTok{ }\DecValTok{100}
\NormalTok{xnew =}\StringTok{ }\KeywordTok{apply}\NormalTok{(M[tr,}\OperatorTok{-}\DecValTok{1}\NormalTok{], }\DecValTok{2}\NormalTok{, }\ControlFlowTok{function}\NormalTok{(X) }\KeywordTok{seq}\NormalTok{(}\KeywordTok{min}\NormalTok{(X), }\KeywordTok{max}\NormalTok{(X), }\DataTypeTok{length.out=}\NormalTok{size))}
\NormalTok{grid =}\StringTok{ }\KeywordTok{expand.grid}\NormalTok{(xnew[,}\DecValTok{1}\NormalTok{], xnew[,}\DecValTok{2}\NormalTok{])}
\NormalTok{grid.yhat =}\StringTok{ }\KeywordTok{knn}\NormalTok{(M[tr,}\OperatorTok{-}\DecValTok{1}\NormalTok{], M[tr,}\DecValTok{1}\NormalTok{], }\DataTypeTok{k=}\NormalTok{k, }\DataTypeTok{test=}\NormalTok{grid)}
\NormalTok{np =}\StringTok{ }\DecValTok{300}
\KeywordTok{par}\NormalTok{(}\DataTypeTok{mar=}\KeywordTok{rep}\NormalTok{(}\DecValTok{2}\NormalTok{,}\DecValTok{4}\NormalTok{), }\DataTypeTok{mgp =} \KeywordTok{c}\NormalTok{(}\DecValTok{1}\NormalTok{, }\DecValTok{1}\NormalTok{, }\DecValTok{0}\NormalTok{))}
\KeywordTok{contour}\NormalTok{(xnew[,}\DecValTok{1}\NormalTok{], xnew[,}\DecValTok{2}\NormalTok{], }\DataTypeTok{z =} \KeywordTok{matrix}\NormalTok{(grid.yhat, size), }\DataTypeTok{levels=}\NormalTok{.}\DecValTok{5}\NormalTok{, }
        \DataTypeTok{xlab=}\KeywordTok{expression}\NormalTok{(}\StringTok{"x"}\NormalTok{[}\DecValTok{1}\NormalTok{]), }\DataTypeTok{ylab=}\KeywordTok{expression}\NormalTok{(}\StringTok{"x"}\NormalTok{[}\DecValTok{2}\NormalTok{]), }\DataTypeTok{axes=}\OtherTok{FALSE}\NormalTok{,}
        \DataTypeTok{main =} \KeywordTok{paste0}\NormalTok{(k,}\StringTok{"-nearest neighbors"}\NormalTok{), }\DataTypeTok{cex=}\FloatTok{1.2}\NormalTok{, }\DataTypeTok{labels=}\StringTok{""}\NormalTok{)}
\KeywordTok{points}\NormalTok{(grid, }\DataTypeTok{pch=}\StringTok{"."}\NormalTok{, }\DataTypeTok{cex=}\DecValTok{1}\NormalTok{, }\DataTypeTok{col=}\NormalTok{grid.yhat)}
\KeywordTok{points}\NormalTok{(M[}\DecValTok{1}\OperatorTok{:}\NormalTok{np,}\OperatorTok{-}\DecValTok{1}\NormalTok{], }\DataTypeTok{col=}\KeywordTok{factor}\NormalTok{(M[}\DecValTok{1}\OperatorTok{:}\NormalTok{np,}\DecValTok{1}\NormalTok{]), }\DataTypeTok{pch =} \DecValTok{1}\NormalTok{, }\DataTypeTok{lwd =} \FloatTok{1.5}\NormalTok{)}
\KeywordTok{legend}\NormalTok{(}\StringTok{"topleft"}\NormalTok{, }\KeywordTok{c}\NormalTok{(}\StringTok{"Player 1 wins"}\NormalTok{, }\StringTok{"Player 2 wins"}\NormalTok{), }
       \DataTypeTok{col=}\KeywordTok{c}\NormalTok{(}\StringTok{"red"}\NormalTok{, }\StringTok{"black"}\NormalTok{), }\DataTypeTok{pch=}\DecValTok{1}\NormalTok{)}
\KeywordTok{box}\NormalTok{()}
\end{Highlighting}
\end{Shaded}

\includegraphics{comp1_files/figure-latex/unnamed-chunk-13-1.pdf} The
strategy for finding K in the above code chunk is finding the greatest K
in which the average K over all folds is less than the minimum K + its
standard deviation.

\textbf{Q14:}

\begin{Shaded}
\begin{Highlighting}[]
\NormalTok{K=}\StringTok{ }\DecValTok{30}
\KeywordTok{library}\NormalTok{(pROC)}
\end{Highlighting}
\end{Shaded}

\begin{verbatim}
## Type 'citation("pROC")' for a citation.
\end{verbatim}

\begin{verbatim}
## 
## Attaching package: 'pROC'
\end{verbatim}

\begin{verbatim}
## The following object is masked from 'package:colorspace':
## 
##     coords
\end{verbatim}

\begin{verbatim}
## The following objects are masked from 'package:stats':
## 
##     cov, smooth, var
\end{verbatim}

\begin{Shaded}
\begin{Highlighting}[]
\CommentTok{# knn with prob=TRUE outputs the probability of the winning class}
\CommentTok{# therefore we have to do an extra step to get the probability of player 1 winning}
\NormalTok{auc =}\StringTok{ }\KeywordTok{rep}\NormalTok{(}\OtherTok{NA}\NormalTok{,K)}
\ControlFlowTok{for}\NormalTok{ (k }\ControlFlowTok{in} \DecValTok{1}\OperatorTok{:}\NormalTok{K)\{}
\NormalTok{  KNNclass=class}\OperatorTok{::}\KeywordTok{knn}\NormalTok{(}\DataTypeTok{train=}\NormalTok{M[tr,}\OperatorTok{-}\DecValTok{1}\NormalTok{], }\DataTypeTok{cl=}\NormalTok{M[tr,}\DecValTok{1}\NormalTok{], }\DataTypeTok{test=}\NormalTok{M[}\OperatorTok{-}\NormalTok{tr,}\OperatorTok{-}\DecValTok{1}\NormalTok{], }\DataTypeTok{k =}\NormalTok{ k,}\DataTypeTok{prob=}\OtherTok{TRUE}\NormalTok{)}
\NormalTok{  KNNprobwinning=}\KeywordTok{attributes}\NormalTok{(KNNclass)}\OperatorTok{$}\NormalTok{prob}
\NormalTok{  KNNprob=}\StringTok{ }\KeywordTok{ifelse}\NormalTok{(KNNclass }\OperatorTok{==}\StringTok{ "0"}\NormalTok{, }\DecValTok{1}\OperatorTok{-}\NormalTok{KNNprobwinning, KNNprobwinning)}
\CommentTok{# now KNNprob has probability that player 1 wins, for all matches in the test set}

\NormalTok{  KNN.roc =}\StringTok{ }\KeywordTok{roc}\NormalTok{(}\DataTypeTok{response =}\NormalTok{M[}\OperatorTok{-}\NormalTok{tr,}\DecValTok{1}\NormalTok{],}\DataTypeTok{predictor =}\NormalTok{ KNNprob)}
\NormalTok{  auc[k] =}\StringTok{ }\KeywordTok{auc}\NormalTok{(KNN.roc)}
\NormalTok{\}}
\KeywordTok{print}\NormalTok{(}\StringTok{"Maximum Area under the curve"}\NormalTok{)}
\KeywordTok{max}\NormalTok{(auc)}
\KeywordTok{print}\NormalTok{(}\StringTok{'Occuring at K = '}\NormalTok{)}
\KeywordTok{which}\NormalTok{(auc }\OperatorTok{==}\StringTok{ }\KeywordTok{max}\NormalTok{(auc))}

\NormalTok{KNNclass=class}\OperatorTok{::}\KeywordTok{knn}\NormalTok{(}\DataTypeTok{train=}\NormalTok{M[tr,}\OperatorTok{-}\DecValTok{1}\NormalTok{], }\DataTypeTok{cl=}\NormalTok{M[tr,}\DecValTok{1}\NormalTok{], }\DataTypeTok{test=}\NormalTok{M[}\OperatorTok{-}\NormalTok{tr,}\OperatorTok{-}\DecValTok{1}\NormalTok{], }\DataTypeTok{k =}\NormalTok{ K,}\DataTypeTok{prob=}\OtherTok{TRUE}\NormalTok{)}
\NormalTok{KNNprobwinning=}\KeywordTok{attributes}\NormalTok{(KNNclass)}\OperatorTok{$}\NormalTok{prob}
\NormalTok{KNNprob=}\StringTok{ }\KeywordTok{ifelse}\NormalTok{(KNNclass }\OperatorTok{==}\StringTok{ "0"}\NormalTok{, }\DecValTok{1}\OperatorTok{-}\NormalTok{KNNprobwinning, KNNprobwinning)}
\CommentTok{# now KNNprob has probability that player 1 wins, for all matches in the test set}
\NormalTok{KNN.roc =}\StringTok{ }\KeywordTok{roc}\NormalTok{(}\DataTypeTok{response =}\NormalTok{M[}\OperatorTok{-}\NormalTok{tr,}\DecValTok{1}\NormalTok{],}\DataTypeTok{predictor =}\NormalTok{ KNNprob)}
\KeywordTok{plot}\NormalTok{(KNN.roc)}
\end{Highlighting}
\end{Shaded}

\includegraphics{comp1_files/figure-latex/unnamed-chunk-14-1.pdf}

\begin{Shaded}
\begin{Highlighting}[]
\KeywordTok{print}\NormalTok{(}\StringTok{"K = 30"}\NormalTok{)}
\KeywordTok{auc}\NormalTok{(KNN.roc)}
\end{Highlighting}
\end{Shaded}

\begin{verbatim}
## [1] "Maximum Area under the curve"
## [1] 0.8200387
## [1] "Occuring at K = "
## [1] 17
## [1] "K = 30"
## Area under the curve: 0.8178
\end{verbatim}

The ROC-curve is found by calculating the sensitivity and specificity of
the model, where all possible values of the decicion thresholds are
being considered. The AUC is the area under the ROC-curve, and a good
classifier has a high value of AUC \(\in[0,1]\). The AUC for K = 30, our
choice from the previous question, is \(0.8178\).

We also found that K = 17 yields the optimal, or largest value for the
AUC.

The interpretation of the AUC is the proportion of time the model ranks
a random positive observation (true positive) higher than a random
negative (false positive) observation.

Random guessing would rank a random positive observation higher than a
negative observation 50\% of the time, and thus the AUC of random
guessing is 0.50.

\textbf{Q15:}

\begin{Shaded}
\begin{Highlighting}[]
\CommentTok{# new classifier y_hat(x) = argmax_k(x_k)}

\NormalTok{new_classifier =}\StringTok{ }\KeywordTok{ifelse}\NormalTok{(M[}\OperatorTok{-}\NormalTok{tr,}\DecValTok{2}\NormalTok{] }\OperatorTok{>}\StringTok{ }\NormalTok{M[}\OperatorTok{-}\NormalTok{tr,}\DecValTok{3}\NormalTok{],}\DecValTok{1}\NormalTok{,}\DecValTok{0}\NormalTok{)}
\NormalTok{argmax =}\StringTok{ }\KeywordTok{as.factor}\NormalTok{(new_classifier)}

\NormalTok{conf_knn =}\StringTok{ }\KeywordTok{confusionMatrix}\NormalTok{(test.pred,Mdf[}\OperatorTok{-}\NormalTok{tr,}\DecValTok{1}\NormalTok{])}
\NormalTok{conf_knn}\OperatorTok{$}\NormalTok{table}

\NormalTok{conf_argmax =}\StringTok{ }\KeywordTok{confusionMatrix}\NormalTok{(argmax,Mdf[}\OperatorTok{-}\NormalTok{tr,}\DecValTok{1}\NormalTok{])}
\NormalTok{conf_argmax}\OperatorTok{$}\NormalTok{table}


\NormalTok{k =}\StringTok{ }\KeywordTok{tail}\NormalTok{(}\KeywordTok{which}\NormalTok{(cv.e }\OperatorTok{<}\StringTok{ }\NormalTok{cv.e[k.min] }\OperatorTok{+}\StringTok{ }\NormalTok{cv.se[k.min]), }\DecValTok{1}\NormalTok{)}
\NormalTok{size =}\StringTok{ }\DecValTok{100}
\NormalTok{xnew =}\StringTok{ }\KeywordTok{apply}\NormalTok{(M[tr,}\OperatorTok{-}\DecValTok{1}\NormalTok{], }\DecValTok{2}\NormalTok{, }\ControlFlowTok{function}\NormalTok{(X) }\KeywordTok{seq}\NormalTok{(}\KeywordTok{min}\NormalTok{(X), }\KeywordTok{max}\NormalTok{(X), }\DataTypeTok{length.out=}\NormalTok{size))}
\NormalTok{grid =}\StringTok{ }\KeywordTok{expand.grid}\NormalTok{(xnew[,}\DecValTok{1}\NormalTok{], xnew[,}\DecValTok{2}\NormalTok{])}
\NormalTok{grid.yhat =}\StringTok{ }\KeywordTok{knn}\NormalTok{(M[tr,}\OperatorTok{-}\DecValTok{1}\NormalTok{], M[tr,}\DecValTok{1}\NormalTok{], }\DataTypeTok{k=}\NormalTok{k, }\DataTypeTok{test=}\NormalTok{grid)}
\NormalTok{np =}\StringTok{ }\DecValTok{300}
\KeywordTok{par}\NormalTok{(}\DataTypeTok{mar=}\KeywordTok{rep}\NormalTok{(}\DecValTok{2}\NormalTok{,}\DecValTok{4}\NormalTok{), }\DataTypeTok{mgp =} \KeywordTok{c}\NormalTok{(}\DecValTok{1}\NormalTok{, }\DecValTok{1}\NormalTok{, }\DecValTok{0}\NormalTok{))}
\KeywordTok{contour}\NormalTok{(xnew[,}\DecValTok{1}\NormalTok{], xnew[,}\DecValTok{2}\NormalTok{], }\DataTypeTok{z =} \KeywordTok{matrix}\NormalTok{(grid.yhat, size), }\DataTypeTok{levels=}\NormalTok{.}\DecValTok{5}\NormalTok{, }
        \DataTypeTok{xlab=}\KeywordTok{expression}\NormalTok{(}\StringTok{"x"}\NormalTok{[}\DecValTok{1}\NormalTok{]), }\DataTypeTok{ylab=}\KeywordTok{expression}\NormalTok{(}\StringTok{"x"}\NormalTok{[}\DecValTok{2}\NormalTok{]), }\DataTypeTok{axes=}\OtherTok{FALSE}\NormalTok{,}
        \DataTypeTok{main =} \KeywordTok{paste0}\NormalTok{(k,}\StringTok{"-nearest neighbors"}\NormalTok{), }\DataTypeTok{cex=}\FloatTok{1.2}\NormalTok{, }\DataTypeTok{labels=}\StringTok{""}\NormalTok{,}\DataTypeTok{col =} \StringTok{'black'}\NormalTok{,}\DataTypeTok{lwd =} \DecValTok{3}\NormalTok{)}
\KeywordTok{points}\NormalTok{(grid, }\DataTypeTok{pch=}\StringTok{"."}\NormalTok{, }\DataTypeTok{cex=}\DecValTok{1}\NormalTok{, }\DataTypeTok{col=}\NormalTok{grid.yhat)}
\KeywordTok{points}\NormalTok{(M[}\DecValTok{1}\OperatorTok{:}\NormalTok{np,}\OperatorTok{-}\DecValTok{1}\NormalTok{], }\DataTypeTok{col=}\KeywordTok{factor}\NormalTok{(M[}\DecValTok{1}\OperatorTok{:}\NormalTok{np,}\DecValTok{1}\NormalTok{]), }\DataTypeTok{pch =} \DecValTok{1}\NormalTok{, }\DataTypeTok{lwd =} \FloatTok{1.5}\NormalTok{)}
\KeywordTok{abline}\NormalTok{(}\DecValTok{0}\NormalTok{,}\DecValTok{1}\NormalTok{,}\DataTypeTok{col=}\StringTok{'blue'}\NormalTok{,}\DataTypeTok{pch=}\StringTok{'-'}\NormalTok{,}\DataTypeTok{lwd =} \DecValTok{3}\NormalTok{)}

\KeywordTok{legend}\NormalTok{(}\StringTok{"topleft"}\NormalTok{, }\KeywordTok{c}\NormalTok{(}\StringTok{"Player 1 wins"}\NormalTok{, }\StringTok{"Player 2 wins"}\NormalTok{,}\StringTok{'argmax class boundary'}\NormalTok{,}\StringTok{'knn class boundary'}\NormalTok{), }
       \DataTypeTok{col=}\KeywordTok{c}\NormalTok{(}\StringTok{"red"}\NormalTok{, }\StringTok{"black"}\NormalTok{,}\StringTok{'blue'}\NormalTok{,}\StringTok{'black'}\NormalTok{), }\DataTypeTok{pch=}\KeywordTok{c}\NormalTok{(}\StringTok{'o'}\NormalTok{,}\StringTok{'o'}\NormalTok{,}\StringTok{'-'}\NormalTok{,}\StringTok{'-'}\NormalTok{))}
\KeywordTok{box}\NormalTok{()}
\end{Highlighting}
\end{Shaded}

\includegraphics{comp1_files/figure-latex/unnamed-chunk-15-1.pdf}

\begin{Shaded}
\begin{Highlighting}[]
\NormalTok{table1 =}\StringTok{ }\NormalTok{conf_knn}\OperatorTok{$}\NormalTok{table}
\NormalTok{table2 =}\StringTok{ }\NormalTok{conf_argmax}\OperatorTok{$}\NormalTok{table}
\NormalTok{knn.misclasserror =}\StringTok{ }\NormalTok{(table1[}\DecValTok{1}\NormalTok{,}\DecValTok{2}\NormalTok{] }\OperatorTok{+}\StringTok{ }\NormalTok{table1[}\DecValTok{2}\NormalTok{,}\DecValTok{1}\NormalTok{])}\OperatorTok{/}\KeywordTok{sum}\NormalTok{(table1)}
\NormalTok{knn.misclasserror}
\NormalTok{argmax.misclasserror =}\StringTok{ }\NormalTok{(table2[}\DecValTok{1}\NormalTok{,}\DecValTok{2}\NormalTok{] }\OperatorTok{+}\StringTok{ }\NormalTok{table2[}\DecValTok{2}\NormalTok{,}\DecValTok{1}\NormalTok{])}\OperatorTok{/}\KeywordTok{sum}\NormalTok{(table2)}
\NormalTok{argmax.misclasserror }
\end{Highlighting}
\end{Shaded}

\begin{verbatim}
##           Reference
## Prediction   0   1
##          0 142  43
##          1  52 157
##           Reference
## Prediction   0   1
##          0 149  47
##          1  45 153
## [1] 0.2411168
## [1] 0.2335025
\end{verbatim}

We see that the argmax classifier yields a lower misclassification error
on the test set, i.e the argmax classifier is preferrable.

\section{Problem 3: Bias-variance
trade-off}\label{problem-3-bias-variance-trade-off}

\textbf{Q16:}

\[
\hat{\boldsymbol{\beta}} = (X^TX)^{-1}X^T\mathbf{Y}
\] Expected value \[
  \begin{split}
    E[\boldsymbol{\hat{\beta}}] &= E[(X^TX)^{-1})X^T\mathbf{Y}]\\
    &\\
    &= (X^TX)^{-1}X^TE[\mathbf{Y}]\\
    &\\
    &= (X^TX)^{-1}X^TE[X\boldsymbol{\beta} + \boldsymbol{\epsilon}]\\
    &\\
    &= (X^TX)^{-1}(X^TX)(E[\boldsymbol{\beta}] + E[\boldsymbol{\epsilon}])\\
    &\\
    &= \boldsymbol{\beta}
  \end{split}
\] Where we are assuming \(\boldsymbol{\epsilon} \sim N(0,I\sigma^2)\).

\[
\begin{split}
\text{Cov}[\boldsymbol{\hat{\beta}}] &= \text{Cov}[(X^TX)^{-1}X^T\mathbf{Y}]\\
&\\
&= (X^TX)^{-1}X^T\text{Cov}[\mathbf{Y}]\bigg((X^TX)^{-1}X^T\bigg)^T\\
&\\
&= (X^TX)^{-1}X^T\sigma^2I\bigg((X^TX)^{-1}X^T\bigg)^T\\
&\\
&= (X^TX)^{-1}(X^TX)(X^TX)^{-1}\sigma^2\\
&\\
&= \sigma^2(X^TX)^{-1}
\end{split}
\] where we have used \(\text{Cov}[\mathbf{Y}] = \sigma^2I\).
\textbf{Q17:} \[
  \begin{split}
    E[\hat{f}(\boldsymbol{x_0})] &= E[\boldsymbol{x_0^T \boldsymbol{\hat{\beta}}}]\\
    &\\
    &= x_0^T E[\boldsymbol{\boldsymbol{\hat{\beta}}}] \\
    &= \boldsymbol{x_0^T}  \boldsymbol{\beta}
  \end{split}
\]

\textbf{Q18:} \[
\text{E}[(Y_0-\hat{f}({\bf x}_0))^2]=[\text{E}(\hat{f}({\bf x}_0)-f({\bf x}_0)]^2+\text{Var}(\hat{f}({\bf x}_0) ) + \text{Var}(\varepsilon)
\]

Ridge estimator: \[
\widetilde{\boldsymbol \beta}=({\bf X}^T{\bf X}+\lambda {\bf I})^{-1}{\bf X}^T{\bf Y}
\]

\textbf{Q19:}

\textbf{Q20:}

\textbf{Q21:}
\[\text{E}[(Y_0-\widetilde{f}({\bf x}_0))^2]=[\text{E}(\widetilde{f}({\bf x}_0)-f({\bf x}_0)]^2+\text{Var}(\widetilde{f}({\bf x}_0) ) + \text{Var}(\varepsilon)\]

\begin{Shaded}
\begin{Highlighting}[]
\NormalTok{values=}\KeywordTok{dget}\NormalTok{(}\StringTok{"https://www.math.ntnu.no/emner/TMA4268/2019v/data/BVtradeoffvalues.dd"}\NormalTok{)}
\NormalTok{X=values}\OperatorTok{$}\NormalTok{X}
\KeywordTok{dim}\NormalTok{(X)}
\NormalTok{x0=values}\OperatorTok{$}\NormalTok{x0}
\KeywordTok{dim}\NormalTok{(x0)}
\NormalTok{beta=values}\OperatorTok{$}\NormalTok{beta}
\KeywordTok{dim}\NormalTok{(beta)}
\NormalTok{sigma=values}\OperatorTok{$}\NormalTok{sigma}
\NormalTok{sigma}
\end{Highlighting}
\end{Shaded}

\begin{verbatim}
## [1] 100  81
## [1] 81  1
## [1] 81  1
## [1] 0.5
\end{verbatim}

Hint: we perform matrix multiplication using \texttt{\%*\%}, transpose
of a matrix \texttt{A} with \texttt{t(A)} and inverse with
\texttt{solve(A)}.

\textbf{Q22:}

\begin{Shaded}
\begin{Highlighting}[]
\NormalTok{sqbias=}\ControlFlowTok{function}\NormalTok{(lambda,X,x0,beta)}
\NormalTok{\{}
\NormalTok{  p=}\KeywordTok{dim}\NormalTok{(X)[}\DecValTok{2}\NormalTok{]}
\NormalTok{  value=}\StringTok{ }\CommentTok{#HERE YOU FILL IN}
\StringTok{  }\KeywordTok{return}\NormalTok{(value)}
\NormalTok{\}}
\NormalTok{thislambda=}\KeywordTok{seq}\NormalTok{(}\DecValTok{0}\NormalTok{,}\DecValTok{2}\NormalTok{,}\DataTypeTok{length=}\DecValTok{500}\NormalTok{)}
\NormalTok{sqbiaslambda=}\KeywordTok{rep}\NormalTok{(}\OtherTok{NA}\NormalTok{,}\KeywordTok{length}\NormalTok{(thislambda))}
\ControlFlowTok{for}\NormalTok{ (i }\ControlFlowTok{in} \DecValTok{1}\OperatorTok{:}\KeywordTok{length}\NormalTok{(thislambda)) sqbiaslambda[i]=}\KeywordTok{sqbias}\NormalTok{(thislambda[i],X,x0,beta)}
\KeywordTok{plot}\NormalTok{(thislambda,sqbiaslambda,}\DataTypeTok{col=}\DecValTok{2}\NormalTok{,}\DataTypeTok{type=}\StringTok{"l"}\NormalTok{)}
\end{Highlighting}
\end{Shaded}

\textbf{Q23:}

\begin{Shaded}
\begin{Highlighting}[]
\NormalTok{variance=}\ControlFlowTok{function}\NormalTok{(lambda,X,x0,sigma)}
\NormalTok{\{}
\NormalTok{  p=}\KeywordTok{dim}\NormalTok{(X)[}\DecValTok{2}\NormalTok{]}
\NormalTok{  inv=}\KeywordTok{solve}\NormalTok{(}\KeywordTok{t}\NormalTok{(X)}\OperatorTok\NormalTok{X}\OperatorTok{+}\NormalTok{lambda}\OperatorTok{*}\KeywordTok{diag}\NormalTok{(p))}
\NormalTok{  value=}\CommentTok{#HERE YOU FILL IN}
\StringTok{  }\KeywordTok{return}\NormalTok{(value)}
\NormalTok{\}}
\NormalTok{thislambda=}\KeywordTok{seq}\NormalTok{(}\DecValTok{0}\NormalTok{,}\DecValTok{2}\NormalTok{,}\DataTypeTok{length=}\DecValTok{500}\NormalTok{)}
\NormalTok{variancelambda=}\KeywordTok{rep}\NormalTok{(}\OtherTok{NA}\NormalTok{,}\KeywordTok{length}\NormalTok{(thislambda))}
\ControlFlowTok{for}\NormalTok{ (i }\ControlFlowTok{in} \DecValTok{1}\OperatorTok{:}\KeywordTok{length}\NormalTok{(thislambda)) variancelambda[i]=}\KeywordTok{variance}\NormalTok{(thislambda[i],X,x0,sigma)}
\KeywordTok{plot}\NormalTok{(thislambda,variancelambda,}\DataTypeTok{col=}\DecValTok{4}\NormalTok{,}\DataTypeTok{type=}\StringTok{"l"}\NormalTok{)}
\end{Highlighting}
\end{Shaded}

\textbf{Q24:}

\begin{Shaded}
\begin{Highlighting}[]
\NormalTok{tot=sqbiaslambda}\OperatorTok{+}\NormalTok{variancelambda}\OperatorTok{+}\NormalTok{sigma}\OperatorTok{^}\DecValTok{2}
\KeywordTok{which.min}\NormalTok{(tot)}
\NormalTok{thislambda[}\KeywordTok{which.min}\NormalTok{(tot)]}
\KeywordTok{plot}\NormalTok{(thislambda,tot,}\DataTypeTok{col=}\DecValTok{1}\NormalTok{,}\DataTypeTok{type=}\StringTok{"l"}\NormalTok{,}\DataTypeTok{ylim=}\KeywordTok{c}\NormalTok{(}\DecValTok{0}\NormalTok{,}\KeywordTok{max}\NormalTok{(tot)))}
\KeywordTok{lines}\NormalTok{(thislambda, sqbiaslambda,}\DataTypeTok{col=}\DecValTok{2}\NormalTok{)}
\KeywordTok{lines}\NormalTok{(thislambda, variancelambda,}\DataTypeTok{col=}\DecValTok{4}\NormalTok{)}
\KeywordTok{lines}\NormalTok{(thislambda,}\KeywordTok{rep}\NormalTok{(sigma}\OperatorTok{^}\DecValTok{2}\NormalTok{,}\DecValTok{500}\NormalTok{),}\DataTypeTok{col=}\StringTok{"orange"}\NormalTok{)}
\KeywordTok{abline}\NormalTok{(}\DataTypeTok{v=}\NormalTok{thislambda[}\KeywordTok{which.min}\NormalTok{(tot)],}\DataTypeTok{col=}\DecValTok{3}\NormalTok{)}
\end{Highlighting}
\end{Shaded}


\end{document}
